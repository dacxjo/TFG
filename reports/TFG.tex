\documentclass[a4paper,10pt]{book}
\usepackage[centertags]{amsmath}
\usepackage{amscd}
\usepackage{amsthm}
\usepackage{amssymb}
\usepackage{enumerate}
\usepackage{multicol}
\usepackage[english,catalan,spanish]{babel}
\usepackage[all]{xy}
\usepackage{color}
\usepackage{tikz}
\usepackage{indentfirst}
\usepackage[utf8]{inputenc}
\usepackage[T1]{fontenc}
\linespread{1.1}
\setlength{\parskip}{10pt}
\usepackage[twoside,bindingoffset=1cm]{geometry}
\usepackage{lmodern}
\usepackage[x11names, dvipsnames, table]{xcolor}
\definecolor{ubblue}{HTML}{0059A2}
\usepackage[colorlinks=true, linkcolor=black, citecolor=ubblue, urlcolor=ubblue]{hyperref}
\usepackage{cleveref}
\usepackage[protrusion=true,expansion=true]{microtype}
\usepackage{cite}


%% Custom packages


%%%%%%%%%%%%%%%%%%%%%%%%%%%%%%%%%%%%%%%%%%%%%%%%%%%%%%%%%%%%%%%%%%%%%%%%%%%
%%%% local definitions for this paper
%%%%%%%%%%%%%%%%%%%%%%%%%%%%%%%%%%%%%%%%%%%%%%%%%%%%%%%%%%%%%%%%%%%%%%%%%%%


%%%%%%%%%%%%%%%%%%%%%% aix{\`o} pels headings %%%%%%%%%%%%%%%%%%%%%%%%
\usepackage{fancyhdr}
\pagestyle{fancy}
\renewcommand{\chaptermark}[1]{\markboth{#1}{}}
\renewcommand{\sectionmark}[1]{\markright{\thesection\ #1}}
\fancyhf{} \fancyhead[LE,RO]{\bfseries\thepage}
\fancyhead[LO]{\bfseries\rightmark} \fancyhead[RE]{\bfseries\leftmark}

\def\paginaenblanc{\newpage%
\thispagestyle{empty}%
\vspace*{2cm}%
\newpage%
\thispagestyle{empty}%
}


%%%%%%%%%%%%%%%%%%%%%%%%%%%%%%%%%%%%%%%%%%%%%%%%%%%%%%%%%%%%%%%%%%%%%%%%%
% aux commands
%%%%%%%%%%%%%%%%%%%%%%%%%%%%%%%%%%%%%%%%%%%%%%%%%%%%%%%%%%%%%%%%%%%%%%%%%
%==========================================================================
% macros to support private authors' notes
%==========================================================================
\newif\ifprivate
\privatetrue
\def\xbar{\vskip0.09in\hrule\vskip0.06in}
\def\private#1{\ifprivate \xbar {\em #1} \xbar
\else \fi}
\def\huh{\ifprivate ??? \marginpar{\Huge ???}
\else \fi}
\def\???{\ifprivate {\bf {???}} \marginpar{\begin{center}{\Huge {\bf ?}}\end{center}}
\else \fi}
%\def\???{\ifprivate {\bf {???}} \marginpar{{\Huge {\bf ?}}}
%\else \fi}
\marginparsep1mm
\def\nota#1{\ifprivate  $\clubsuit$ \marginpar{\parbox[t]{2.4cm}{\begin{center}\tiny #1\end{center}}}
\else \fi}
\def\comment#1{\ifprivate \marginpar{\parbox[t]{2.4cm}{\begin{center}\tiny #1\end{center}}}
\else \fi}
%\def\nota#1{\ifprivate  $\clubsuit$ \marginpar{\parbox[t]{1.8cm}{\tiny #1}}
%\else \fi}
\def\privateeject{\ifprivate\eject\fi}
%\def\???{{\bf {???}} \marginpar{{\Huge {\bf ?}}} }
%%%%%%%%%%%%%%%%%%%%%%%%%%%%%%%%%%%%%%%%%%%%%%%%%%%%%%%%%%%%%%%%%%%%%%%%%%

%%%%%%%%%%%%%%%%%%%%%%%%%%%%%%%%%%%%%%%%%%%%%%%%%%%%%%%%%%%%%%%%%%%%%%%%
%%%%%%%%%%%%%%%%%%%%%%%%%%%%%%%%%%%%%%%%%%%%%%%%%%%%%%%%%%%%%%%%%%%%%%%%
\begin{document}

\pagestyle{empty}

\begin{titlepage}
	\begin{center}
		\begin{figure}[htb]
			\begin{center}
				\includegraphics[width=6cm]{assets/ub_color.pdf}
			\end{center}
		\end{figure}
		
		\def\worktitle{Development of an AI-Based Tool for Molecular Subtype Classification of Invasive Ductal Breast Carcinoma Using Mammography}
		
		\textbf{\LARGE Treball final de grau} \\
		\vspace*{.5cm}
		\textbf{\LARGE GRAU D'ENGINYERIA INFORM\`{A}TICA } \\
		\vspace*{.5cm}
		\textbf{\LARGE Facultat de Matem\`atiques i Inform\`atica\\ Universitat de Barcelona} \\
		\vspace*{1.0cm}
		\rule{16cm}{0.1mm}\\
		\begin{Huge}
			\textbf{Development of an AI-Based Tool for Molecular Subtype Classification of Invasive Ductal Breast Carcinoma Using Mammography} \\
		\end{Huge}
		\rule{16cm}{0.1mm}\\
		
		\vspace{1cm}
		
		\begin{flushright}
			
			
			\vspace*{2.5cm}
			
			\hfill
			
			\renewcommand{\arraystretch}{1.5}
			\begin{tabular}{ll}
				\textbf{\small Autor:}       & \textbf{\small David Bland\'on T\'orrez }                             \\
				\textbf{\small Director:}    & \textbf{\small Dr. Oliver D\'iaz Montesdeoca }                        \\
				\textbf{\small Realitzat a:} & \textbf{\small  Departament de Matem\`{a}tiques i  Inform\`{a}tica  } \\
				\textbf{\small Barcelona,}   & \textbf{\small \today }                                               
			\end{tabular}
			
		\end{flushright}
		
	\end{center}
	
\end{titlepage}

%%%%%%%%%%%%%%%%%%%%%%%%%%%%%%%%%%%%%%%%%%%%%%%%%%%%%%%%%%%%%%%%%%%%%%%%%
\newpage
\selectlanguage{english}
\noindent \textbf{\large Abstract}

// TODO

%%%%%%%%%%%%%%%%%%%%%%%%%%%%%%%%%%%%%%%%%%%%%%%%%%%%%%%%%%%%%%%%%%%%%%%%%

%%%%%%%%%%%%%%%%%%%%%%%%%%%%%%%%%%%%%%%%%%%%%%%%%%%%%%%%%%%%%%%%%%%%%%%%%
\newpage
\selectlanguage{spanish}
\noindent \textbf{\large Resumen}

// TODO

%%%%%%%%%%%%%%%%%%%%%%%%%%%%%%%%%%%%%%%%%%%%%%%%%%%%%%%%%%%%%%%%%%%%%%%%%

%%%%%%%%%%%%%%%%%%%%%%%%%%%%%%%%%%%%%%%%%%%%%%%%%%%%%%%%%%%%%%%%%%%%%%%%%
\newpage
\selectlanguage{catalan}
\noindent \textbf{\large Resum}

// TODO

%%%%%%%%%%%%%%%%%%%%%%%%%%%%%%%%%%%%%%%%%%%%%%%%%%%%%%%%%%%%%%%%%%%%%%%%%
\newpage
\selectlanguage{english}
\noindent \textbf{\large Acknowledgements}

// TODO
%%%%%%%%%%%%%%%%%%%%%%%%%%%%%%%%%%%%%%%%%%%%%%%%%%%%%%%%%%%%%%%%%%%%%%%%%
\selectlanguage{english}
\pagenumbering{roman} \setcounter{page}{0}
\let\cleardoublepage\clearpage
\tableofcontents
\newpage \thispagestyle{empty}
%%%%%%%%%%%%%%%%%%%%%%%%%%%%%%%%%%%%%%%%%%%%%%%%%%%%%%%%%%%%%%%%%%%%%%%%%

\pagestyle{fancy}
\markboth{Introducción}{Introducción}
\newpage \thispagestyle{empty}
%%%%%%%%%%%%%%%%%%%%%%%%%%%%%%%%%%%%%%%%%%%%%%%%%%%%%%%%%%%%%%%%%%%%%%%%%
\mainmatter
\chapter{Introduction}
\section{Problem Context}

Breast cancer has become one of the leading causes of mortality among women and represents the type of cancer with the highest incidence in this population. It is estimated that, on average, one in twenty women worldwide will be diagnosed with this disease during her lifetime \cite{kim_global_2025}. Recent projections suggest that, if the current trend continues, by 2050 approximately 3.2 million new cases and 1.1 million deaths associated with this pathology will be recorded, with a particularly significant impact in countries with a low Human Development Index (HDI) \cite{kim_global_2025}.

In this context, early-diagnosis techniques and tools play a fundamental role in improving patients’ prognosis and survival \cite{wang_early_2017}. However, breast cancer is a heterogeneous disease\footnote{Cellular diversity present within a tumor (intratumoral heterogeneity) or between different tumors in the same individual (intertumoral heterogeneity).} that can be classified into various subtypes according to clinical and—especially—molecular characteristics. The 2013 St. Gallen International Consensus guidelines \cite{goldhirsch_personalizing_2013} recognize four main subtypes based on hormone receptors (estrogen, progesterone) and the proliferation marker Ki67: Luminal A, Luminal B, HER2-positive (HER2-enriched) and Triple Negative (see Figure \ref{fig:subtypes}). This classification has direct clinical implications, as prognosis, therapy response and treatment options largely depend on the molecular subtype to which the tumor belongs.

Currently, molecular characterization of the tumor is carried out mainly through biopsy, an invasive and costly procedure that sometimes must be repeated, which can delay the start of treatment and increase the clinical, physical and emotional burden on patients. Therefore, there is a growing need to develop non-invasive, accessible and efficient methods that can perform this task reliably. In this regard, mammography stands out as a key tool, as it is non-invasive, low-cost and widely used in the early diagnosis of breast cancer.

\begin{figure}
	\centering
	\includegraphics[width=0.8\linewidth]{reports/assets/subtypes.png}
	\caption{The 4 molecular subtypes of breast cancer and their prevalence percentage \cite{harnessing_2024}}
	\label{fig:subtypes}
\end{figure}

In recent years, advances in artificial intelligence (AI), together with the increasing availability of data and ever more efficient computational capacity, have driven the development of deep-learning (DL) models for tasks such as classification, detection and prediction of breast cancer, as well as other diseases. Several studies have shown that these systems can match and even surpass the performance of human experts or computer-aided diagnosis (CAD) systems in such tasks \cite{mckinney_international_2020,pattanaik_breast_2022,meenalochini_deep_2024,hussain_performance_2025}, demonstrating the significant impact of this technology and its potential benefit for clinical practice and patient well-being.

Recently, the classification of molecular subtypes from mammographic images has been explored. Mota et al. (2024) \cite{mota_breast_2024} addressed this task, achieving a 60.62\% AUC in multiclass classification with a ResNet101 architecture. In turn, Rabah et al. (2025) \cite{ben_rabah_multimodal_2025} reached an AUC of 63.79\% using an Xception model and further proposed a multimodal approach that integrated clinical metadata, raising performance to 88.87\% AUC. Although the results obtained in unimodal scenarios are still modest and remain below the clinically useful threshold (\textasciitilde80\% AUC), these studies demonstrate the potential of imaging as a diagnostic source and reinforce the need to continue research in this direction to improve the accuracy and clinical utility of such models.

This study proposes a unimodal approach based exclusively on mammograms from the public CMMD dataset (The Chinese Mammography Database) \cite{cai_online_2023}, with the aim of comparing the performance of state-of-the-art Transformer architectures such as Vision Transformer (ViT), Shifted-Window Transformer (SwinT) and Multi-Axis Vision Transformer (MaxViT) against a traditional deep convolutional neural-network (CNN) baseline. Although multimodal models usually achieve better results by integrating complementary clinical data, focusing solely on mammographic images has high practical value, especially in resource-limited settings or where the standardization of clinical data is not guaranteed. Recent studies have shown that Transformers outperform CNNs in accuracy and robustness in medical-classification tasks thanks to their self-attention mechanisms, which capture global spatial relationships within the image \cite{mauricio_comparing_2023}. Based on these design advantages, this study hypothesizes that Transformer architectures could achieve superior performance in molecular-subtype classification even under a unimodal approach.

Ultimately, this work seeks to contribute to the development of non-invasive diagnostic tools by systematically evaluating Transformer models, in order to advance toward automated, accessible and efficient molecular characterization of breast cancer—especially in contexts where biopsy is not an immediate option—thereby helping to improve equitable access to diagnosis and to reduce therapeutic-intervention times.

\section{Objectives}

The main objective of this study is to develop a deep learning model capable of classifying molecular subtypes of breast cancer using only mammographic images from the \textit{Chinese Mammography Database} (CMMD), without relying on clinical metadata or auxiliary annotations. This task poses significant challenges, such as class imbalance between molecular subtypes, scarcity of labeled data, and the absence of region-of-interest annotations in the images, which hinder effective learning and may limit AI model performance in medical applications.

To address these challenges, the performance of various modern Transformer-based architectures, specifically, Vision Transformer (ViT), Shifted-Window Transformer (SwinT), and Multi-Axis Vision Transformer (MaxViT), will be systematically evaluated and compared to traditional convolutional neural network models. The focus on these architectures lies in their ability to capture global relationships and complex patterns in images through attention mechanisms, which is particularly relevant for identifying subtle features associated with molecular heterogeneity in mammograms.

Extensive experiments will be conducted, incorporating adaptive data augmentation strategies (rotations, flips, and intensity transformations specific to mammography), oversampling techniques, and weighted loss functions to tackle class imbalance. Performance will be evaluated using robust metrics such as weighted F1-score, area under the ROC curve (AUC-ROC), and balanced accuracy, enabling an objective comparison across architectures.

With the results obtained, a comparative analysis will be carried out against previous studies that have addressed the same problem.

Additionally, an explainability analysis of the best-performing model will be conducted through the generation of attention maps and Gradient-weighted Class Activation Mapping (Grad-CAM) techniques. This will help identify the mammographic regions that contribute significantly to the model's decisions, facilitating clinical validation and providing insights into the visual patterns associated with each molecular subtype.

... // TODO: Develop conclusion and statistical analysis (p-values)

\section{Planning}

\subsection{Tasks to Develop}

// TODO

\subsection{Schedule}

// TODO

%%%%%%%%%%%%%%%%%%%%%%%%%%%%%%%%%%%%%%%%%%%%%%%%%%%%%%%%%%%%%%%%%%%%%%%%%

\chapter{Background}
\section{Breast Cancer}
\section{Molecular Subtypes}
\section{Screening Process}
\section{Mammography}

\chapter{Technology Review}

\chapter{Materials and Methods}
\section{CMMD Dataset}
\section{Image Preprocessing}

\chapter{Results and Discussion}
\section{Results}

\chapter{Conclusions and Future Work}

%%%%%%%%%%%%%%%%%%%%%%%%%%%%%%%%%%%%%%%%%%%%%%%%%%%%%%%%%%%%%%%%%%%%%%%%%
\backmatter
\selectlanguage{english}
\addcontentsline{toc}{chapter}{Bibliography}
\bibliographystyle{ieeetr}
\bibliography{references}

\end{document}